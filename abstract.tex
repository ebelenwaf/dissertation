% abstract.tex (Abstract)

\addcontentsline{toc}{chapter}{Abstract}

\chapter*{Abstract}
The Internet of Things (IoT) offers immense benefits by
enabling devices to leverage networked resources thereby making intelligent
decisions. The numerous heterogeneous connected devices that exist throughout
the IoT system creates new security and privacy concerns. Some of these concerns can
be overcome through data trust, transparency, and integrity, which can be
achieved with data provenance. Data provenance, also known as data lineage, provides a complete history of transformations that occurs on a data object from the time it was created to its current state. Data provenance has been explored in the areas of scientific computing, business, forensic analysis, and intrusion detection. Data provenance can help in detecting and mitigating malicious cyber attacks. 


 \par This dissertation explores the integration of data provenance within the IoT ecosystem by taking an in-depth look at the collection and adoption of provenance in mitigating malicious intrusion. The first step towards provenance adoption is achieved by developing Provenance-Aware IoT (PA-IoT), a provenance collection tool for IoT devices.This framework generates provenance data through application manual instrumentation with low execution overhead. With the integration of PA-IoT, a fundamental problem arises; provenance has the possibility to generate tremendous amount of data. We address the storage challenge of provenance collection by adopting a graph summarization approach which allows grouping of similar graphs into super nodes. 


%policy approach for provenance collection and storage which allows flexibility in deciding what provenance data to keep. 


\textcolor{red}{We evaluate the effectiveness of our framework by looking at an application of provenance data using an intrusion detection system, which detects malicious threats against IoT applications.}


