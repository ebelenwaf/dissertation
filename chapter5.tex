% chap5.tex (Definitions, Theorem and Proof)

%\chapter{CPS Device Sensor Data Anomaly Detection using Provenance Graphs}
\chapter{Anomaly Detection using Provenance Graphs}
\section{Provenance-Based Anomaly Detection} \label{sec:prov_anomaly}
This section discusses the application of provenance data for intrusion detection of malicious events in a CPS device. 

%We propose an approach to identifying anomalous events using provenance graphs. This approach involves the use of a similarity metric to compare observed provenance graphs with provenance graphs derived from an application's normal execution. The result is an anomaly score which is  compared with a previously set threshold to classify an observed provenance graph as either anomalous or benign. We evaluate the effectiveness of our approach with a sample IoT application that simulates a climate control system.

\section{Introduction}

IoT devices have become an essential part of our daily lives in commercial, industrial, and infrastructure systems. These devices offer benefits to consumers by interacting with the physical environment through sensors and actuators, which allows device automation thereby improving efficiency. Unfortunately, the proliferation of IoT devices has led to an increase in the number of remotely exploitable vulnerabilities leading to new attack vectors that could have disastrous financial and physical implications. In 2015, security researchers demonstrated a vulnerability on the Jeep vehicle which allowed remote control of the automotive system over the Internet \cite{jeep_vulnerabilty}. In 2016, researchers discovered a vulnerability that allows Internet connected smart thermostats to be subject to remote ransomware attacks in which an attacker gains sole control of the thermostat until a fee is paid \cite{smart_thermostat}. These are a few examples of some of the potential malicious vulnerabilities that could have devastating, long-lasting impact on an IoT system. \par Intrusion detection \cite{Lazarevic2005} is the process of discovering malicious exploits in a system. One way of detecting an intrusion is by the use of anomaly detection techniques. An anomaly, also referred to as an outlier, is data that deviates from the normal system behavior. Anomalies could be indicative of a system fault or that a system has been compromised by a malicious event. Due to the sensitive nature of safety critical systems, detecting malicious attacks is of utmost importance.

 \par We propose an approach to identifying anomalous sensor events using provenance graphs. This approach involves the use of a similarity metric to compare observed provenance graphs with provenance graphs derived from an application's normal execution. The result is an anomaly score which is  compared with a previously set threshold to classify an observed provenance graph as either anomalous or benign. We evaluate the effectiveness of our approach with a sample IoT application that simulates a climate control system.
 
 \par Our method of comparing the similarity of provenance graphs was inspired by an information retrieval technique for document retrieval. Given a corpus $D = \{ d_1,..., d_n\}$, and query, $q$, how do we find document(s) $\{d_x,....d_y\}$ which are similar to $q$ and rank them by order of importance. To achieve this, documents contained in the corpus are converted into a vector space representation which allows documents to be ranked based on some similarity metric.


\subsection{Graph Similarity} \label{similarity}
Similarity is a measure of how identical two objects are, for example, by measuring the angle between objects (using cosine similarity) or a linear distance (using euclidean distance) between the objects. In this work, we use cosine similarity as our similarity metric. This was inspired by the use of information retrieval techniques for query ranking. e.g., given a corpus $D = \{ d_1,..., d_n\}$, and query, $q$, how do we find document(s) $\{d_x,....d_y\}$ which are similar to $q$ and rank them by order of importance. Cosine similarity is a measure of orientation between two non-zero vectors. It measures the cosine of the angle between the vectors. Two vectors which are at an angle of 90$\degree$ have a similarity of 0, two vectors which are identical (with an angle of 0\degree) have a cosine of 1, and two vectors which are completely opposite (with an angle of 180\degree) have a similarity of -1. Since we are concerned with the similarity between vectors, we are only concerned with the positive values bounded in [0,1]. The cosine similarity between two vectors, $X$ and $Y$, is computed by:

\[\mathbf{\cos{(\theta)}} = \dfrac{X \cdot  Y}{ \lVert \mathbf{X} \rVert \cdot \lVert \mathbf{Y} \rVert} =\dfrac{\sum_{i}^n X_i Y_i }{\sqrt[]{\sum_{i}^n X_i^2} \times \sqrt[]{\sum_{i}^n Y_i^2}}  \]

%Algorithm \ref{alg:graph_similarity} describes how to calculate the cosine similarity of two graphs $p_x, p_y$ given their vector representations $u_x, u_y$

%Each graph consist of an edge list which contains all edges contained in the provenance graph.

\par In order to apply cosine similarity between provenance graphs, we compute a vector representation which reduces the graph into an $n$-dimensional vector space where $n$ represents the total number of edges contained in the union of all edge sets. Figure \ref{prov_vector} illustrates the vector space conversion of provenance graphs. $\boldsymbol{G_1}$, and $\boldsymbol{G_2}$ which consists of vertices $A,B,E,F,G, I, J, S, R$ and edge labels $aOBO, wAW, wGB, wDB$. The vector space representation of $u_1$ is the occurrence of edges contained in the edge set of graph $G_1$, which  are also found in the collective union of edge sets. Algorithm \ref{graph_to_vector} further outlines the concept of graph to vector conversion.

\subsection{Anomaly Detection on Provenance Graphs}
Anomaly detection involves the use of rule-based, statistical, clustering or classification techniques to determine normal or anomalous data instances. The process of determining all anomalous instances in a given dataset is a complex task. A major challenge in anomaly detection is providing the right feature set from the data to use for detection. Another challenge exists in defining what constitutes as normal system behavior. Most anomaly detection using point-based data often fail to include the dependencies that exist between data points. 

\par Due to the ubiquitous nature of CPS devices, there are a wide array of vulnerabilities associated with them. In designing our anomaly detection framework, we expect an attacker's footprint is reflected through the data flow as depicted in the provenance graph. Our algorithm detects attacks such as false data injection, and state change as depicted in information flow of sensor events in provenance graphs.

\begin{algorithm}[h!]  
\caption{Graph to vector conversion.} 
 \label{graph_to_vector} 
\begin{algorithmic}[1]

\Procedure{GraphtoVector}{$E, E_G$}

\State $n \gets |E_G|$

\State $Q[k],Q[i] \gets 0, 0 \leq i < n$

\For{$e_j \in E$}

%\State \textit{Found} $\gets$ \textbf{False}

\For{$e_g \in E_G \mid 0 \leq g < n$}

%\If{$e_j \sim e_g$}
\If{$e_j = e_g$}
\State $Q[g] \gets Q[g] + 1$
%\State \textit{Found} $\gets$ \textbf{True}
\EndIf

\EndFor

\EndFor


\State \textbf{return} $Q$

\EndProcedure

\end{algorithmic}
\end{algorithm}


\begin{algorithm}[h!]

\caption{Detection algorithm given an observation phase graph set, $P$, a detection phase graph, $G$, and a threshold $T$.} 
 \label{alg:graph_anomaly} 

\begin{algorithmic}[1]  

\Procedure{GraphAnomaly}{$P,G,T$}

\State \textbf{INPUT: } $P=\{G_0,...,G_n\} \mid G_i\gets (V_i, E_i), 0 \leq i \leq n.$

%\State $E_G \gets \{\}$
%\For{$p_i = (V_i, E_i) \in P, 0 \leq i \leq n$}
\State $E_G \gets \cup_{i=0}^{n} E_i$

\State $Q \gets GraphtoVector(G, E_G)$ 

\State $Z \gets \{\}$

\For{$G_i \in P$}
\State $N_i \gets GraphtoVector(G_i, E_G)$

\State $z \gets Cosine\_Similarity(Q, N_i)$

\State $Z \gets Z \cup z_i$

\EndFor	

\State $s_{val} \gets max(Z)$

\If{ $s_{val} \geq T$}
\State \textbf{return} normal

\EndIf

\State \textbf{return} anomaly
\EndProcedure
\end{algorithmic}
\end{algorithm}

\par Many CPS devices implement a control systems in which sensor data is used as an input in a feedback loop to an actuator. The operations of most control systems are regular and predictable. For example, in a thermostat application, temperature readings generated might be converted from Celsius to Fahrenheit and utilized as feedback to an actuator. Each iteration of a control loop sequence generates a path in a provenance graph. This notion can be leveraged to define an expected provenance graph for each application. 

\par The expected regularity of provenance graphs in CPS applications motivates a supervised learning approach to anomaly detection. This approach consists of two phases: observation phase, also known as the training phase, and the detection or test phase. In the observation phase, the system collects provenance data considered to be a representation of the normal system behavior. In the detection phase, the provenance graph set is compared with the provenance graph derived from subsequent observations to determine if an anomaly exists by measuring similarity between this graph and the provenance graph set. Note that provenance graphs from the observation and detection phase form a graph set. A global edge set, $E_G$ represents the union of edge sets contained in a graph set. Algorithm \ref{alg:graph_anomaly} is the graph anomaly detection function given an observation phase graph set, $P$, and a detection phase graph, $G$. $Z$ represents a list of the cosine scores from comparing each of the provenance graphs in the observation phase graph set with a detection phase graph. The maximum cosine similarity score of elements contained in $Z$ is taken as the score for the detection phase graph, and if that score is above the threshold $T$ then the graph is considered normal, otherwise it is classifed an anomaly.

\subsubsection{Defining Anomaly Threshold}

An anomaly threshold $T$ is a score that defines at what point a provenance graph contained in the test data is considered anomalous. Ensuring a proper threshold score is used for detection is an important task that requires extensive knowledge of the application domain. The threshold often is manually set to a value that is defined by domain experts. For automatic anomaly threshold detection, one can use prediction methods to define an anomaly score. Threshold values can be increased or decreased to alter the anomaly detection accuracy.




%\begin{algorithm}
%\caption{Cosine similarity given two vectors.}
% \label{alg:graph_similarity}
% 
%\begin{algorithmic}[1]
%
%\Procedure{sim}{$u_x,u_y$}
%\State \textbf{INPUT: } $u_x=\{u_{x_0},...,u_{x_n}\},u_y=\{u_{y_0},...,u_{y_n}\}$
%\State $S_p, S_{u_x}, S_{u_y} \gets$ 0
%%\State $S_{ux} \gets$ 0
%%\State $S_{uy} \gets$ 0
%\For{$ 0 \leq i \leq n$}
%
%\State $S_p \gets S_p + (u_x[i] \times u_y[i])$
%\State $ S_{u_x} \gets  S_{u_x} + u_x[i]^2$
%\State $S_{u_y} \gets S_{u_y} + u_y[i]^2$
%
%\EndFor
%
%\State $G \gets \dfrac{S_p}{\sqrt[]{S_{u_x}} \times \sqrt[]{S_{u_y}} }$
%
%%\State $G \gets S_p \div \sqrt[]{S_{ux} \times S_{uy}}$
%
%\State \textbf{return} $G$
%
%\EndProcedure
%
%\end{algorithmic}
%
%\end{algorithm}
%
%\begin{algorithm}
%\caption{Algorithm to construct the global edge set}
%
%\begin{algorithmic}[1]  
%\caption{Construction of global edge set from a set of provenance graphs.}
%\label{graph_to_globaledgelist}
%
%\Procedure{GraphSetToGlobalEdgeSet}{$P$}
%\State \textbf{INPUT: } $P=\{p_0,...,p_n\} \mid p_i\gets (V_i, E_i), 0 \leq i \leq n.$
%
%%\State $E_G \gets \{\}$
%%\For{$p_i = (V_i, E_i) \in P, 0 \leq i \leq n$}
%\State $E_G \gets \cup_{i=0}^{n} E_i$
%
%%\EndFor
%\State \textbf{return} $E_G$
%
%\EndProcedure
%
%\end{algorithmic}
%\end{algorithm}
%
%
%\begin{algorithm}[h!]  
%\caption{Graph to vector conversion.} 
% \label{graph_to_vector} 
%\begin{algorithmic}[1]
%
%\Procedure{GraphtoVector}{$E, E_G$}
%
%\State $n \gets |E_G|$
%
%\State $Q[k],Q[i] \gets 0, 0 \leq i < n$
%
%\For{$e_j \in E$}
%
%%\State \textit{Found} $\gets$ \textbf{False}
%
%\For{$e_g \in E_G \mid 0 \leq g < n$}
%
%%\If{$e_j \sim e_g$}
%\If{$e_j = e_g$}
%\State $Q[g] \gets Q[g] + 1$
%%\State \textit{Found} $\gets$ \textbf{True}
%\EndIf
%
%\EndFor
%
%\EndFor
%
%
%\State \textbf{return} $Q$
%
%\EndProcedure
%
%\end{algorithmic}
%\end{algorithm}
%
%\begin{algorithm}[h!] 
%\caption{Calculates an anomaly score given a set of cosine similarity values.}  
% \label{alg:min_func} 
%\begin{algorithmic}[1]  
%
%\Procedure{calculateAnomalyScore}{$Z$}
%\State \textbf{INPUT: } $Z=\{z_0,...,z_n\} , 0 \leq i \leq n.$
%\State score $\gets 0.0$
%\For{$z_i \in Z$}
%\If{score $> z_i$}
%\State score $\gets z_i$
%\EndIf
%\EndFor	
%\State \textbf{return} score
%\EndProcedure
%
%\end{algorithmic}
%\end{algorithm}
%
%
%\begin{algorithm}[h!]
%
%\caption{Detection algorithm given an observation phase graph set, $P$, a detection phase graph, $p$, and a $threshold$.} 
% \label{alg:graph_anomaly} 
%
%\begin{algorithmic}[1]  
%
%\Procedure{GraphAnomaly}{$P,p,threshold$}
%
%\State $E_G \gets GraphSetToGlobalEdgeSet(P \cup p)$
%
%\State $Q \gets GraphtoVector(p, E_G)$ 
%
%\State $Z \gets \{\}$
%
%\For{$p_i \in P$}
%\State $N_i \gets GraphtoVector(p_i, E_G)$
%
%\State $z \gets SIM(Q, N_i)$
%
%\State $Z \gets Z \cup z_i$
%
%\EndFor	
%
%\State $s_{val} \gets calculateAnomalyScore(Z)$
%
%\If{ $s_{val} \geq$ threshold}
%\State \textbf{return} normal
%
%\EndIf
%
%\State \textbf{return} anomaly
%\EndProcedure
%
%
%
%\end{algorithmic}
%
%
%
%\end{algorithm}



%\section{Summary and Conclusion}
%
%In this paper, we propose an anomaly detection algorithm for detecting anomalous instances of sensor based events in an IoT device using provenance graphs. We evaluate our approach with a preliminary study on an IoT application which simulates a climate control system. In the future, we plan on conducting further experimentation to identify the false and true positive rates of our algorithm using select IoT application dataset. 




%\section{Acknowledgment}
%This research has been supported in part by US National Science Foundation (CNS grant No. 1646317). Any opinions, findings and conclusions or recommendations expressed in this material are those of the author(s) and do not necessarily reflect the views of NSF.




