% chap6.tex (Significance and Future Work)
\chapter{Conclusion and Future Directions}


\section{Summary}
This dissertation focuses on the integration of data provenance within CPS ecosystem for provenance-based anomaly detection. To achieve this, we developed a provenance-aware framework that collects the information flow of data dependencies within a CPS device in a non-intrusive fashion. Our approach leverages the PROV-DM ontology for portability and uses a software-based tracing mechanism to collect provenance without requiring operating system intervention. We address the issue of storage overhead by proposing a data pruning heuristic at the trace level which eliminates irrelevant data not required by our provenance intrusion detection system. 



\section{Future Research Direction}

Some of the proposed future work are listed as follows:

\begin{enumerate}

\item \textbf{Security and Privacy protection of Provenance Data:} Provenance collection raises privacy issues. How do we ensure that the vast amount of data collected is not invasive to the privacy of device users. Privacy preserving techniques can be used to anonymize provenance data. Proper encryption and authentication techniques \cite{Hasan:2009:CFP:1525908.1525909} are needed to ensure the confidentiality, and integrity of provenance data.

\item \textbf{Real-time Anomaly Detection:} We provide an offline anomaly detection using provenance graphs. Future work will include the modification of our IDS algorithm to include detection of anomalous instances in real-time. 

%\item Determine the maximum capacity based on the desired capacity of a provenance application

%\item \textbf{Blockchain based Provenance} With the  advent IoT, there is an advantage in which decentralized processing system offers. One research direction is to explore how to integrate data provenance using blockchain's distributed ledger system.

\item \textbf{A hybrid approach to storage optimization of trace data:} We will explore how the use of a combination of techniques such as compression and graph summarization in addition to our approach will further reduce the size of trace data generated. 

\item \textbf{End to End integration of Prov-CPS:} Current implementation of Prov-CPS does not include an automated flow between various components. A more seamless framework is needed where all of the components of PROV-CPS operates without user interference.

\item \textbf{Anomaly detection using a machine learning approach:} Machine learning algorithms have been shown to effectively detect anomalous instances however, most of these algorithms are computationally intensive and incur additional memory overhead. It will be beneficial to explore the use of advanced machine learning algorithms such as neural networks, and support vector machines for anomaly detection using provenance data generated from our framework. In addition, it is important to determine what the maximum memory capacity is for integrating the aforementioned algorithms on CPS devices.


%\item {Neural network integration with Prov-CPS}

\item\textbf{CPS-Cloud integration} This dissertation focuses on provenance collection on end devices. In the future, we plan to explore the integration of provenance in the cloud. Specifically, we will explore how data is disseminated across end devices to the cloud and how this interaction can be further streamlined for optimal service execution.

\end{enumerate}

